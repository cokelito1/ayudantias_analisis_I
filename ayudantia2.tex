\documentclass[a4paper,oneside,10.5pt]{USMArt}

\usepackage{personal}
\usepackage{comment}

\usepackage{biblatex}
\addbibresource{bibliografia.bib}
\nocite{*}

\usepackage[letterpaper, left=2cm, right=2cm]{geometry}

\title{Ayudantía 2 - Analisis I}
\sigla{MAT-225 }
\ramo{Analisis I}
\profesor{Alberto Mercado}
\semestre{2025-1}
\author{Jorge Bravo}

\begin{document}

\maketitle

\begin{prob}
  Muestre que $\mathcal{C}([0, 1])$ no es completo con la metrica inducida por la norma
  \begin{equation*}
    ||f|| = \sqrt{\int_{0}^{1} f(x)^{2} dx}
  \end{equation*}
\end{prob}

\begin{sol}
  Consideremos la sucesion de funciones definida por
  \begin{gather*}
    f_{n} : [0, 1] \to \RR\\
    x \mapsto \begin{cases}
      0 & \text{ si } x \in [0, \frac{1}{2} - \frac{1}{n}]\\
      n(x - (\frac{1}{2} - \frac{1}{n})) & x \in (\frac{1}{2} - \frac{1}{n}, \frac{1}{2})\\
      1 & \text{ si } x \in [\frac{1}{2}, 1]
        \end{cases}
  \end{gather*}

  Veamos que esta sucesion es de Cauchy, supongamos sin perdida de generalidad que $n < m$
  \begin{align*}
    ||f_{n} - f_{m}||^{2} &= \int_{0}^{1} (f_{n} - f_{m})(x)^{2} dx\\
                          &= \int_{\frac{1}{2} - \frac{1}{n}}^{\frac{1}{2} - \frac{1}{m}} n^{2}(x - (\frac{1}{2} - \frac{1}{n}))^{2} dx + \int_{\frac{1}{2} - \frac{1}{m}}^{\frac{1}{2}} (n(x - (\frac{1}{2} - \frac{1}{n})) - m(x - (\frac{1}{2} - \frac{1}{m})))^{2} dx\\
                          &\leq \frac{n^{2}}{3}(x - (\frac{1}{2} - \frac{1}{n}))^{3}|_{\frac{1}{2} - \frac{1}{n}}^{\frac{1}{2} - \frac{1}{m}} + \frac{1}{m} \cdot \max_{x \in [\frac{1}{2} - \frac{1}{m}, \frac{1}{2}]} (n(x - (\frac{1}{2} - \frac{1}{n})) - m(x - (\frac{1}{2} - \frac{1}{m})))^{2}
  \end{align*}

  Al calcular la derivada de la funcion dentro del maximo obtenemos
  \begin{equation*}
    2(n(x - (\frac12 - \frac{1}{n})) - m(x - (\frac{1}{2} - \frac{1}{m}))) \cdot (n - m)
  \end{equation*}

  Notemos que el factor de la derecha es negativo, mientras que el factor de la izquierda es una recta que vale $0$ en
  $\frac{1}{2}$ y que en $0$ vale $2(1 - \frac{n}{2} + \frac{m}{2} - 1) = m - n$ que es positivo. Por lo tanto la recta
  tiene pendiente negativa y dado que en $\frac{1}{2}$ vale $0$, tenemos que en $[\frac{1}{2} - \frac{1}{m}, \frac{1}{2}]$ es positiva, por lo tanto la funcion dentro del maximo es decreciente y podemos evaluar en el extremo
  izquierdo para obtener el maximo

  \begin{align*}
    ||f_{n} - f_{m}||^{2} &\leq \frac{n^{2}}{3}(\frac{1}{n} - \frac{1}{m})^{3} + \frac{1}{m} \cdot (n(\frac{1}{n} - \frac{1}{m}))^{2}\\
                          &= \frac{n^{2}}{3}(\frac{1}{n} - \frac{1}{m})^{3} + \frac{1}{m} \cdot (1 - \frac{n}{m})^{2}\\
                          &= \frac{n^{2}}{3}(\frac{1}{n} - \frac{1}{m})^{3} + \frac{1}{m} \cdot \frac{(m - n)^{2}}{m^{2}}\\
                          &\leq \frac{n^{2}}{3n^{3}} + \frac{1}{m}\\
                          &= \frac{3}{n} + \frac{1}{m} \to 0
  \end{align*}
\end{sol}

Por lo tanto tenemos que es de Cauchy. Veamos que no puede converger dentro del espacio. Supongamos que existe $f \in \mathcal{C}([0, 1])$ tal que $f_{n} \to f$, luego tenemos que
\begin{equation*}
  ||f_{n} - f||^{2} \to 0
\end{equation*}

En particular, tenemos que
\begin{equation*}
  \int_{0}^{\frac{1}{2} - \frac{1}{n}} f(x)^{2} dx + \int_{\frac{1}{2} - \frac{1}{n}}^{\frac{1}{2}} (f(x) - f_{n}(x))^{2} dx + \int_{\frac{1}{2}}^{1}(f(x) - 1)^{2} dx = 0
\end{equation*}

Dado que son números positivos las 3 integrales, cada una por separado tienen que irse a $0$. De la ultima integral
obtenemos que $f(x)|_{[\frac{1}{2}, 1]} \equiv 1$, por la continuidad de $f$.

Notemos que del teorema fundamental del calculo, la funcion definida por
\begin{equation*}
  y \mapsto \int_{0}^{\frac{1}{2} - y} f(x)^{2} dx
\end{equation*}

Es diferenciable, en particular continua. Por lo tanto
\begin{equation*}
  \lim_{n \to \infty} \int_{0}^{\frac{1}{2} - \frac{1}{n}} f(x)^{2} dx = 0 \implies \int_{0}^{\frac{1}{2}} f(x)^{2} dx = 0
\end{equation*}

Por lo tanto $f(x)|_{[0, \frac{1}{2}]} \equiv 0$, esto es una contradiccion pues entonces la funcion $f$ tiene que tomar
$2$ valores en $\frac{1}{2}$.

\newpage

El objetivo de esta ayudantía es explorar todos los conceptos que se han introducido en el curso hasta el
momento. Esto lo haremos investigando los números p-adicos los cuales forman un espacio (ultra)-métrico
con varias propiedades poco intuitivas.

\begin{defi}{(Valor absoluto)}
  Sea $A$ un dominio integral, decimos que $|\cdot| : A \to \RR$ es valor absoluto sobre $A$ si cumple
  \begin{enumerate}
    \item $|x| \geq 0$
    \item $|x| = 0 \iff x = 0$
    \item $|xy| = |x| \cdot |y|$
    \item $|x + y| \leq |x| + |y|$
  \end{enumerate}
\end{defi}

\begin{obs}
  En nuestro caso el anillo (cuerpo) sobre el que trabajaremos sera $\QQ$, por lo que podemos reemplazar eso en la
  definición si no sabe lo que es un dominio integral.
\end{obs}

\begin{teo}
  Sea $|\cdot|$ un valor absoluto sobre $A$, luego

  \begin{gather*}
    d : A \times A \to \RR\\
    (x, y) \mapsto |x - y|
  \end{gather*}

  es una metrica
\end{teo}
\begin{proof}
  Similar a la demotraci\'on que una norma induce una métrica
\end{proof}

\begin{defi}{(Valuación p-adica sobre $\ZZ$)}
  Sea $p$ un numero primo, definimos la valuación p-adica sobre $\ZZ$ como la funci\'on
  \begin{gather*}
    v_{p} : \ZZ \to \ZZ \cup \{\infty\}\\
    n \mapsto v_{p}(n)
  \end{gather*}

  Donde $v_{p}(n)$ cumple que $x = p^{v_{p}(n)}x' $ y $v_{p}(n)$ no divide a x'. Definimos $v_{p}(0) = \infty$
\end{defi}
\begin{obs}
  La valuaci\'on p-adica nos dice que tan divisible es $x$ por $p$.
\end{obs}

\begin{ejemplo}
  Sea $p = 2$, luego tenemos que
  \begin{gather*}
    v_{p}(8) = 3\\
    v_{p}(9) = 0\\
    v_{p}(10) = 1
  \end{gather*}
\end{ejemplo}

\begin{prop}
  Sea $a, b \in \ZZ$, entonces
  \begin{equation*}
    v_{p}(ab) = v_{p}(a) + v_{p}(b)
  \end{equation*}
\end{prop}

\begin{proof}
  Notemos que el resultado es trivial si $a$ o $b$ son $0$. Sean $a, b \in \ZZ^{>0}$, luego tenemos que
  \begin{gather*}
    a = p^{v_{p}(a)}a'\\
    b = p^{v_{p}(b)}b'
  \end{gather*}

  Por lo tanto
  \begin{equation*}
    ab = p^{v_{p}(a) + v_{p}(b)}a'b'
  \end{equation*}

  Dado que $p$ no divide a $a'$ ni a $b'$, entonces no divide a $a'b'$.
  Con lo que tenemos

  \begin{equation*}
    v_{p}(ab) = v_{p}(a) + v_{p}(b)
  \end{equation*}
\end{proof}

\begin{defi}{(Valuaci\'on p-adica sobre $\QQ$)}
  Sea $p$ un primo, definimos la valuaci\'on p-adica sobre $\QQ$ como
  \begin{gather*}
    v_{p} : \QQ \to \ZZ \cup \{\infty\}\\
    \frac{a}{b} \mapsto v_{p}(a) - v_{p}(b)
  \end{gather*}
\end{defi}

\begin{prop}
  La valuaci\'on p-adica sobre $\QQ$ esta bien definida
\end{prop}
\begin{proof}
  Supongamos que $\frac{a}{b} = \frac{c}{d}$ con $a, b \neq 0$, luego notemos que $ad = bc$, entonces
  \begin{equation*}
    v_{p}(a) + v_{p}(d) = v_{p}(ad) = v_{p}(bc) = v_{p}(b) + v_{p}(c)
  \end{equation*}

  Por lo tanto
  \begin{equation*}
    v_{p}(a) - v_{p}(b) = v_{p}(c) - v_{p}(d)
  \end{equation*}
\end{proof}

\begin{prop}
  Sea $\frac{a}{b}, \frac{c}{d} \in \QQ$, entonces $v_{p}(\frac{ac}{bd}) = v_{p}(\frac{a}{b}) + v_{p}(\frac{c}{d})$
\end{prop}
\begin{proof}
  Notemos que
  \begin{equation*}
    v_{p}(\frac{ac}{bd}) = v_{p}(ac) - v_{p}(bd) = v_{p}(a) + v_{p}(c) - v_{p}(b) - v_{p}(d) = v_{p}(\frac{a}{c}) + v_{p}(\frac{b}{d})
  \end{equation*}
\end{proof}

\begin{prop}
  Sea $n, m \in \ZZ$, entonces tenemos que
  \begin{equation*}
    v_{p}(n + m) \geq \min \{v_{p}(n), v_{p}(m)\}
  \end{equation*}
\end{prop}

\begin{proof}
  Si $n + m = 0$ el resultado es trivial, al igual que si $n$ o $m$ son $0$. Por lo tanto supongamos que
  $n + m \neq 0, n \neq 0, m \neq 0$. Por lo tanto tenemos que
  \begin{align*}
    n &= p^{v_{p}(n)}n'\\
    m &= p^{v_{p}(m)}m'
  \end{align*}

  Sea $v = \min \{v_{p}(n), v_{p}(m)\}$. Luego se tiene que
  \begin{equation*}
    n + m = p^{v_{p}(n)}n' + p^{v_{p}(m)}m' = p^{v}(p^{v_{p}(n) - v}n' + p^{v_{p}(m) - v}m')
  \end{equation*}

  Por lo tanto se tiene el resultado, pues si $p^{v_{p}(n) - v}n' + p^{v_{p}(m) - v}m'$ no es divisble por $p$ tenemos
  que $v_{p}(n + m) = v$, si $p^{v_{p}(n) - v}n' + p^{v_{p}(m) - v}m'$ es divisble por $p$, entonces al factorizar esos
  $p$, tendremos que $v_{p}(n + m) > v$.
\end{proof}

\begin{prop}
  Sea $\frac{a}{b}, \frac{c}{d} \in \QQ$, entonces
  \begin{equation*}
    v_{p}(\frac{a}{b} + \frac{c}{d}) \geq \min \{v_{p}(\frac{a}{b}), v_{p}(\frac{c}{d})\}
  \end{equation*}
\end{prop}
\begin{proof}
  El caso en que $\frac{a}{b} + \frac{c}{d} = 0$ es trivial, por lo tanto supondremos
  que $\frac{a}{b} + \frac{c}{d} \neq 0$. Supongamos además que $v_{p}(\frac{a}{b}) \leq v_{p}(\frac{c}{d})$,
  luego $\frac{a}{b} < \infty$, pues si fuera $\infty$ tendríamos $\frac{a}{b} + \frac{c}{d} = 0$, si $\frac{c}{d} = 0$ entonces
  el resultado es trivial por lo tanto supondremos $\frac{c}{d} \neq 0$.

  Notemos que
  \begin{equation*}
    v_{p}(\frac{a}{b} + \frac{c}{d}) = v_{p}(\frac{ad + bc}{bd}) = v_{p}(ad + bc) - v_{p}(bd)
  \end{equation*}

  Por la proposición anterior $v_{p}(ad + bc) \geq \min \{v_{p}(ad), v_{p}(bc)\}$ supongamos que el mínimo es $v_{p}(ad)$,
  entonces tenemos
  \begin{equation*}
    v_{p}(\frac{a}{b} + \frac{c}{d}) \geq v_{p}(ad) - v_{p}(bd) = v_{p}(a) + v_{p}(d) - v_{p}(b) - v_{p}(d) = v_{p}(\frac{a}{b})
  \end{equation*}

  Si el minimo es $v_{p}(bc)$ entonces tenemos que
  \begin{equation*}
    v_{p}(\frac{a}{b} + \frac{c}{d}) \geq v_{p}(bc) - v_{p}(bd) = v_{p}(b) + v_{p}(c) - v_{p}(b) -v_{p}(d) = v_{p}(\frac{c}{d}) \geq v_{p}(\frac{a}{b})
  \end{equation*}

  Por lo que se tiene el resultado.
\end{proof}

\begin{defi}
  Sea $p$ un primo, Se define el valor absoluto p-adico sobre $\QQ$ como
  \begin{equation*}
    |x|_{p} = \begin{cases}
      p^{-v_{p}(x)} & \text{ si } x \neq 0\\
      0 & \text{ si } x = 0
      \end{cases}
  \end{equation*}
\end{defi}

\begin{prop}
  El valor absoluto p-adico es un valor absoluto sobre $\QQ$.
\end{prop}

\begin{proof}
  Las 2 primeras condiciones son directas de la definici\'on, veamos la condición $3$.
  Si $x = 0$ o $y = 0$ el resultado es trivial. Sea $x, y \in \QQ$ luego tenemos que
  \begin{equation*}
    |xy|_{p} = p^{-v_{p}(xy)} = p^{-v_{p}(x) - v_{p}(y)} = p^{-v_{p}(x)}p^{-v_{p}(y)} = |x|_{p} \cdot |y|_{p}
  \end{equation*}

  Ahora demostraremos algo mas fuerte que la desigualdad triangular usual, mostraremos que
  \begin{equation*}
    |x + y|_{p} \leq \max \{|x|_{p}, |y|_{p}\}
  \end{equation*}

  Si $x + y = 0$ el resultado es trivial. Por lo tanto supongamos que $x + y \neq 0$, sin perdida de generalidad
  supondremos que $|x|_{p} \geq |y|_{p}$, se sigue de esto que $x \neq 0$. Dado que
  $p^{-v_{p}(x)} = |x|_{p} \geq |y|_{p} = p^{-v_{p}(y)}$ se tiene que $v_{p}(x) \leq v_{p}(y)$. De la proposición $5$
  tenemos que $v_{p}(x + y) \geq \min \{v_{p}(x), v_{p}(y)\} = v_{p}(x)$, por lo tanto tenemos que
  \begin{equation*}
    p^{-v_{p}(x + y)} \leq p^{-v_{p}(x)}
  \end{equation*}
  Con lo que se tiene el resultado pues $\max \{|x|_{p}, |y|_{p}\} = |x|_{p}$.
\end{proof}

\begin{obs}
  Dado que tenemos un valor absoluto sobre $\QQ$, este define una métrica, analizaremos como es esta métrica
\end{obs}

\begin{ejemplo}
  Calculemos $|\frac{75}{73}|_{5}$, notemos que
  \begin{equation*}
    v_{p}(\frac{75}{73}) = v_{p}(75) - v_{p}(73) = v_{p}(5^{2}\cdot3) - v_{p}(73) = 2
  \end{equation*}

  Pues 73 es primo. Luego $|\frac{75}{73}|_{5} = 5^{-2} = \frac{1}{25}$
\end{ejemplo}

\begin{ejemplo}
  Calculemos $d_{11}(89, 4082)$ donde $d_{11}$ es la métrica inducida por $|\cdot|_{11}$
  \begin{equation*}
    d_{11}(89, 4082) = |4082 - 89|_{11} = |3993|_{11} = |11^{3} \cdot 3|_{11} = 11^{-3}
  \end{equation*}
\end{ejemplo}

\begin{prop}
  Sea $\QQ$ con la métrica inducida por $|\cdot|_{p}$, entonces
  \begin{equation*}
    \lim_{n \to \infty} p^{n} = 0
  \end{equation*}
\end{prop}

\begin{proof}
  Calculemos $d(p^{n}, 0)$
  \begin{equation*}
    d(p^{n}, 0) = |p^{n}|_{p} = p^{-n}
  \end{equation*}

  Dado que $p^{-n} \to 0$ en $\RR$, tenemos el resultado.
\end{proof}
\begin{obs}
  En algún sentido la sucesión $p^{n}$ en $\QQ$ con esta métrica se parece a la sucesión $\frac{1}{n}$ en $\RR$,
  pues ambas son sucesiones de términos positivos que se acercan a $0$.
\end{obs}

\begin{obs}
  Tal como al completar $\QQ$ con la métrica usual se obtiene $\RR$, se puede completar $\QQ$ con la métrica p-adica y
  se obtienen los espacios métricos $\QQ_{p}$ los cuales son completos y $\QQ$ son densos en el. Esto se
  hace mediante un proceso llamado completacion que veremos en otra ayudantía. Por lo tanto en $\QQ_{p}$ una sucesión
  converge si y solo si es de Cauchy.
\end{obs}

\begin{prop}
  Sea $(a_{n})$ una sucesión en $\QQ_{p}$, entonces $(a_{n})$ es de Cauchy si y solo si para todo $\varepsilon > 0$,
  existe $N \in \NN$ tal que si $n > \NN$ entonces $d_{p}(x_{n}, x_{n + 1}) < \varepsilon$
\end{prop}

\begin{proof}
  La implicancia de izquierda a derecha es trivial. Supongamos que se tiene lo de la derecha. Luego se $\varepsilon > 0$
  notemos que
  \begin{equation*}
    |a_{n} - a_{m}|_{p} = |(a_{n} - a_{n + 1}) + (a_{n + 1} - a_{n +2}) + ... + (a_{m - 1} - a_{m})|_{p} \leq \max \{|a_{n} -a_{n + 1}|_{p}, ..., |a_{m - 1} - a_{m}|_{p}\}
  \end{equation*}

  Tomemos $N$ tal que si $n > N$ entonces $|a_{n} - a_{n + 1}|_{p} < \varepsilon$, luego cada uno de los elementos del
  máximo es menor que $\varepsilon$ si $n, m > N$, por lo que tenemos
  \begin{equation*}
    |a_{n} - a_{m}|_{p} < \varepsilon
  \end{equation*}
\end{proof}
\begin{adv}
  Esto es falso en la mayoría de espacios métricos, por ejemplo considere $a_{n} = \ln(n)$ como una sucesión de $\RR$,
  luego tenemos que
  \begin{equation*}
    |\ln(n) - \ln(n + 1)| = |\ln(\frac{n}{n + 1})| \to 0
  \end{equation*}

  Pero no es de Cauchy pues $\ln(n)$ no converge.
\end{adv}

\begin{prop}
  Sea $(a_{n})$ una sucesión en $\QQ_{p}$ entonces $\sum_{n = 1}^{\infty} a_{n}$ converge si y solo si $\lim_{n \to \infty} a_{n} = 0$
\end{prop}

\begin{proof}
  De izquierda a derecha se deja como ejercicio. De derecha a izquierda usamos la proposición $8$, sea $\varepsilon > 0$
  dado que $\lim_{n \to \infty} a_{n} = 0$, existe $N$ tal que si $n > N$ entonces $|a_{n}|_{p} < \varepsilon$. Notemos que
  si $n > N$ entonces
  \begin{equation*}
    |S_{n + 1} - S_{n}| = |a_{n + 1}| < \varepsilon
  \end{equation*}

  Por lo tanto la sucesión es de Cauchy y por tanto converge pues $\QQ_{p}$ es completo.
\end{proof}
\begin{adv}
  Esto es falso en general, pues en $\RR$ la sucesión $\frac{1}{n}$ tiene por limite $0$ pero su serie no converge.
\end{adv}

\begin{prop}
  La serie $\sum_{n = 0}^{\infty} p^{n}$ converge en $\QQ_{p}$ y su suma es $\frac{1}{1 - p}$.
\end{prop}
\begin{proof}
  El hecho que converga sigue directo de la proposición $7$ y $9$. Sabemos que las sumas parciales son de la forma
  \begin{equation*}
    S_{n} = \frac{1 - p^{n}}{1 - p}
  \end{equation*}

  Luego tenemos que
  \begin{equation*}
    d(S_{n}, \frac{1}{1 - p}) = |\frac{1}{1 - p} - \frac{1 - p^{n}}{1 - p}|_{p} = |\frac{p^{n}}{1 - p}|_{p} = |p^{n}|_{p} \cdot |\frac{1}{1 - p}|_{p} = p^{-n} \cdot 1 \to 0
  \end{equation*}

  Por lo tanto converge.
\end{proof}

\begin{ejemplo}
  Sea $p = 2$, aplicando la proposición anterior tenemos que
  \begin{equation*}
    \sum_{n = 0}^{1} 2^{n} = \frac{1}{1 - 2} = -1
  \end{equation*}
\end{ejemplo}

\printbibliography
\end{document}
